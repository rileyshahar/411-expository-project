% vim:ft=tex
% \documentclass[sigplan,screen,nonacm,natbib=false,balance=false,11pt]{acmart}
\documentclass[11pt]{article}

% macros
% vim:ft=tex

\usepackage{amsthm, amssymb, amsmath}
\usepackage[unicode=true, pdfusetitle,
	bookmarks=true,bookmarksnumbered=false,
	breaklinks=false,
	backref=false,
	colorlinks=true,
	linkcolor=blue,
	citecolor=blue,
	urlcolor=blue,
	final
]{hyperref}
\usepackage[margin=1in]{geometry}

% formatting
\usepackage{enumitem}
\usepackage{multicol}

% fonts
\usepackage{bbm}

% figs
\usepackage{float}
\usepackage{tikz-cd}
\usepackage{subcaption}
\usepackage{mathpartir}
\usepackage{adjustbox}

% refs
\usepackage[
	backend=biber,
	style=alphabetic
]{biblatex}
\addbibresource{references.bib}
\usepackage[capitalise]{cleveref}

% quality of life
\renewcommand{\emptyset}{\varnothing}

% diagrams
\newcommand{\punctuation}[1]{\makebox[0pt][l]{#1}}

% font conventions
\newcommand{\cat}[1]{{\mathcal{#1}}}
\newcommand{\scat}[1]{{\textsc{\text{#1}}}}

% operators
\DeclareMathOperator{\Hom}{Hom}
\DeclareMathOperator{\Mod}{Mod}

% theorems
\theoremstyle{plain}
\newtheorem{thm}{Theorem}
\newtheorem{lemma}[thm]{Lemma}
\newtheorem{prop}[thm]{Proposition}
\newtheorem{cor}{Corollary}[thm]

\theoremstyle{definition}
\newtheorem{dfn}[thm]{Definition}
\newtheorem{ex}[thm]{Example}

\theoremstyle{remark}
\newtheorem*{rmk}{Remark}
\newtheorem*{ntn}{Notation}

% alphabets
\def\cA{\mathcal{A}}
\def\cB{\mathcal{B}}
\def\cC{\mathcal{C}}
\def\cD{\mathcal{D}}
\def\cE{\mathcal{E}}
\def\cF{\mathcal{F}}
\def\cG{\mathcal{G}}
\def\cH{\mathcal{H}}
\def\cI{\mathcal{I}}
\def\cJ{\mathcal{J}}
\def\cK{\mathcal{K}}
\def\cL{\mathcal{L}}
\def\cM{\mathcal{M}}
\def\cN{\mathcal{N}}
\def\cO{\mathcal{O}}
\def\cP{\mathcal{P}}
\def\cQ{\mathcal{Q}}
\def\cR{\mathcal{R}}
\def\cS{\mathcal{S}}
\def\cT{\mathcal{T}}
\def\cU{\mathcal{U}}
\def\cV{\mathcal{V}}
\def\cW{\mathcal{W}}
\def\cX{\mathcal{X}}
\def\cY{\mathcal{Y}}
\def\cZ{\mathcal{Z}}

\def\AA{\mathbb{A}}
\def\BB{\mathbb{B}}
\def\CC{\mathbb{C}}
\def\DD{\mathbb{D}}
\def\EE{\mathbb{E}}
\def\FF{\mathbb{F}}
\def\GG{\mathbb{G}}
\def\HH{\mathbb{H}}
\def\II{\mathbb{I}}
\def\JJ{\mathbb{J}}
\def\KK{\mathbb{K}}
\def\LL{\mathbb{L}}
\def\MM{\mathbb{M}}
\def\NN{\mathbb{N}}
\def\OO{\mathbb{O}}
\def\PP{\mathbb{P}}
\def\QQ{\mathbb{Q}}
\def\RR{\mathbb{R}}
\def\SS{\mathbb{S}}
\def\TT{\mathbb{T}}
\def\UU{\mathbb{U}}
\def\VV{\mathbb{V}}
\def\WW{\mathbb{W}}
\def\XX{\mathbb{X}}
\def\YY{\mathbb{Y}}
\def\ZZ{\mathbb{Z}}

\def\kk{\Bbbk}

\def\fA{\mathfrak{A}}
\def\fB{\mathfrak{B}}
\def\fC{\mathfrak{C}}
\def\fD{\mathfrak{D}}
\def\fE{\mathfrak{E}}
\def\fF{\mathfrak{F}}
\def\fG{\mathfrak{G}}
\def\fH{\mathfrak{H}}
\def\fI{\mathfrak{I}}
\def\fJ{\mathfrak{J}}
\def\fK{\mathfrak{K}}
\def\fL{\mathfrak{L}}
\def\fM{\mathfrak{M}}
\def\fN{\mathfrak{N}}
\def\fO{\mathfrak{O}}
\def\fP{\mathfrak{P}}
\def\fQ{\mathfrak{Q}}
\def\fR{\mathfrak{R}}
\def\fS{\mathfrak{S}}
\def\fT{\mathfrak{T}}
\def\fU{\mathfrak{U}}
\def\fV{\mathfrak{V}}
\def\fW{\mathfrak{W}}
\def\fX{\mathfrak{X}}
\def\fY{\mathfrak{Y}}
\def\fZ{\mathfrak{Z}}

\def\fa{\mathfrak{a}}
\def\fb{\mathfrak{b}}
\def\fc{\mathfrak{c}}
\def\fd{\mathfrak{d}}
\def\fe{\mathfrak{e}}
\def\ff{\mathfrak{f}}
\def\fg{\mathfrak{g}}
\def\fh{\mathfrak{h}}
% \def\fi{\mathfrak{i}}
\def\fj{\mathfrak{j}}
\def\fk{\mathfrak{k}}
\def\fl{\mathfrak{l}}
\def\fm{\mathfrak{m}}
\def\fn{\mathfrak{n}}
\def\fo{\mathfrak{o}}
\def\fp{\mathfrak{p}}
\def\fq{\mathfrak{q}}
\def\fr{\mathfrak{r}}
\def\fs{\mathfrak{s}}
\def\ft{\mathfrak{t}}
\def\fu{\mathfrak{u}}
\def\fv{\mathfrak{v}}
\def\fw{\mathfrak{w}}
\def\fx{\mathfrak{x}}
\def\fy{\mathfrak{y}}
\def\fz{\mathfrak{z}}




% metadata
\title{Category $\cO$}
\author{Riley Shahar}
\date{}
% \orcid{0009-0003-9485-4399}
% \affiliation{
% 	\institution{Reed College}
% 	\city{Portland}
% 	% \state{Oregon}
% 	\country{USA}
% }
% \email{rileyshahar@reed.edu}

% \keywords{TODO}

\begin{document}
\maketitle

\begin{abstract}
	[abstract goes here]
\end{abstract}

\section{Introduction}

 [Introduction goes here.] We largely follow \cite{humphreys} and \cite[Chapter
	19]{carter}.

In \Cref{sec:preliminaries}, we recall some basic notions from categorical
algebra, in particular the notion of an abelian category. We define category
$\cO$ in \Cref{sec:definition} and explore its basic categorical properties. In
\Cref{sec:decomposition}, we study the \emph{block decomposition} of $\cO$ into
blocks defined by central characters. We then study the action of
\emph{translation functors} on these blocks in \Cref{sec:translation functors}.
\Cref{sec:conclusion} concludes.

\section{Preliminaries}
\label{sec:preliminaries}

\subsection{Data}

We fix throughout a finite-dimensional semisimple complex Lie algebra $\fg$ with
Cartan subalgebra $\fh$. We choose roots $R$ and positive roots $R^+$, inducing
a decomposition
\begin{equation}\label{eqn:cartan decomposition}
	\fg = \fn^-\oplus\fh\oplus\fn =
	\bigoplus_{\alpha<0}\fg_\alpha\oplus\fh\oplus\bigoplus_{\alpha>0}\fg_\alpha
\end{equation}
and Borel subalgebra $\fb = \fh\oplus\fn$. As usual, we denote by $U\fg$ and
$Z\fg$ the universal enveloping algebra of $\fg$ and its center, respectively.

\subsection{Abelian Categories}
\label{sec:abelian categories}

The category $\scat{Grp}$ of groups is surprisingly hard to work with. In
particular, we cannot quotient by an arbitrary subgroup: instead, we can only
quotient by normal subgroups. Category-theoretically, where we prefer to work
with morphisms, the central difficulty is that given a subgroup, it does not
necessarily emerge as the kernel of a homomorphism. In contrast, the category
$\scat{Ab}$ of abelian groups is extremely nice. Because any abelian group is
normal, we can take a quotient by any subgroup in $\scat{Ab}$, and
correspondingly, any subgroup emerges as the kernel of some homomorphism.

Similarly, the category $\scat{Ring}$ of rings is not as nice as we would like:
we can only quotient by ideals, rather than arbitrary subrings. Even the
category $\scat{CRing}$ of commutative rings is not good enough. Instead, we
prefer to work in the category $\Mod R$ of modules over a ring $R$. Any subring
of $R$ is a module of $R$, and we can take a quotient of any two modules and get
another module, so any subring has a quotient---and hence is the kernel of some
morphism---in $\Mod R$.

The notion of an \emph{abelian category} formalizes the properties needed to do
categorical algebra, and in particular to have well-behaved quotients. A
definition can be found in any standard text on categorical algebra, such
as~\cite{borceux}. In lieu of reproducing it here, we state the two most
important properties.

First, each hom-set $\Hom(A, B)$ is an abelian group\footnote{
	In categorical language, we say that an abelian category is \emph{enriched} in
	the category \scat{Ab} of abelian groups. An \scat{Ab}-enriched category is
	also called an \emph{additive} category.
}. This means that we can add morphisms with the same domain and codomain. Even
better, we can combine abelian categories with the same operations we use to
combine abelian groups. For example, given two abelian categories $\cat{A}$ and
$\cat{B}$, we can form their direct sum $\cat{A}\oplus\cat{B}$, whose objects
are pairs $(A,B)$ of objects in $\cat{A}$ and $\cat{B}$, and whose hom-sets are
given by \[ \Hom((A,B), (A',B')) = \Hom(A,A')\oplus\Hom(B,B'). \]

Second, abelian categories are have extremely well-behaved kernels and
cokernels. In particular, given any morphism $f: A\rightarrow B$, we can find a
universal morphism $\ker f: K\rightarrow A$ which factors through $f$ as the
zero map. Conversely, \emph{every} subobject\footnote{The notion of
	\emph{subobject} is the categorical generalization of subset, subgroup, etc.} is
the kernel of some other morphism---this categorically represents the intuitive
idea that every subgroup of an abelian group is normal.

% Third, products and coproducts exist and coincide in abelian categories.

\subsection{Noetherian and Artinian Categories}

Consider some ascending chain of inclusions \[
	X_1\hookrightarrow X_2\hookrightarrow X_3\hookrightarrow\cdots
\]
of submodules of a module (or subobjects of an object) $X$. We say $X$ is
\emph{Noetherian} if any such sequence stabilizes. A category is
Noetherian\footnote{
	For foundational reasons, we also ask that the category is \emph{essentially
		small}, meaning it is equivalent to a category whose morphisms form a set. We
	do not worry about such issues here; all of our categories are essentially
	small. See~\cite[Section 1.6]{maclane} for more details.
} if each object is
Noetherian.

Dually, an object is \emph{Artinian} if any descending chain of
inclusions stabilizes, and a category is Artinian if each object is Artinian.

\section{Basic Properties}
\label{sec:definition}

\subsection{Definition}

We can now define our main object:
\begin{dfn}[category $\cO$]\label{def:category O}
	Category $\cO$ consists of all $U\fg$-modules $V$ which are
	\begin{enumerate}[label={\Roman*}., ref={\Roman*}]
		\item \emph{finitely generated}: there is a finite subset of $V$ whose
		      $U\fg$-orbit is $V$;\label{ax:finitely generated}
		\item \emph{$\fh$-semisimple}: $\fh$ acts semisimply on
		      $V$;\label{ax:h-semisimple}
		\item \emph{locally $\fn$-finite}: each $U\fn$-orbit is finite
		      dimensional.\label{ax:n-finite}
	\end{enumerate}
	Morphisms in $\cO$ are $U\fg$-module homomorphisms\footnote{
		In categorical language, we say that $\cO$ is a \emph{full subcategory} of
		$\Mod U\fg$, indicating that \[\Hom_\cO(V, W) = \Hom_{\Mod U\fg}(V, W)\] for
		any objects $V,W$ of $\cO$.
	}.
\end{dfn}

A few remarks on the definition. Finite generation does not imply that $V$ is
finite-dimensional as a vector space, since the $U\fg$-orbit of an element may
leave its $\CC$-span. The assertion that $\fh$ acts semisimply states exactly
that\[ V = \bigoplus_{\lambda\in\fh^*}V_\lambda, \]i.e. that $V$ is a weight
module. Furthermore, each weight space in $V$ is in fact finite-dimensional, as
$U\fn$ spans only a finite dimension while $U\fh$ fixes this space. As a
consequence, the $Z\fg$-span of any vector in $V$ is finite-dimensional, since
it is contained in a sum of weight spaces. In short, the modules in $\cO$
are \emph{very} nice.

	[Discuss that Verma modules and finite-dimensional modules are in $\cO$.]

\subsection{Categorical Properties}

We now consider the basic categorical properties of $\cO$.

Since $\Mod U\fg$ is Noetherian, $\cO$ is also Noetherian. It follows that $\cO$
is closed under quotients, submodules, and finite direct sums. For example, in
the case of a submodule $W$ of a module $V\in\cO$, observe that $\fh$ still acts
semisimply on $W$, while $U\fn$-orbits are identical to the orbits in $V$, and
so finite-dimensional. Finite-generation is preserved because $\cO$ is
Noetherian---if $W$ were not finitely generated, it would yield an infinite
chain of strict submodules of $V$.

We now observe that $\cO$ is abelian---in fact, by a standard result, the above
closure properties are sufficient for a subcategory of an abelian category to
remain abelian.

Furthermore, $\cO$ is closed under finite tensor products. Finite generation and
local $\fn$-finiteness are preserved because the tensor is finite. Letting $V
	\cong \bigoplus_{\lambda\in\fh^*}V_\lambda$ and $W \cong
	\bigoplus_{\mu\in\fh^*}W_\mu$, we have
\begin{align*}
	V\otimes W & \cong (\bigoplus_{\lambda\in\fh^*}
	V_\lambda)\otimes(\bigoplus_{\mu\in\fh^*} W_\mu)                                                     \\
	           & \cong \bigoplus_{\lambda,\mu\in\fh^*}(V_\lambda\otimes W_\mu)                           \\
	           & \cong \bigoplus_{\nu\in\fh^*}\bigoplus_{\lambda + \mu = \nu} (V_\lambda \otimes W_\mu).
\end{align*}
Since $V_\lambda\otimes W_\mu\subseteq (V\otimes W)_{\lambda + \mu}$, now \[
	\bigoplus_{\lambda + \mu = \nu} (V_\lambda \otimes W_\mu) \subseteq (V\otimes W)_\nu,
\]and so up to isomorphism, \[
	V\otimes W\subseteq \bigoplus_{\nu\in\fh^*} (V\otimes W)_\nu\subseteq V\otimes
	W.
\]Because all the inclusions involved are natural in $V$ and $W$, isormorphism
holds\footnote{
	We remark that this proof, which follows~\cite[Lemma 19.1(ii)]{carter},
	applies in the much more general setting of abelian categories whose objects
	break down into blocks indexed by an abelian group, in this case $\fh^*$.
	Indeed, categorically, the proof amounts to applying naturality of the
	adjunction between $\otimes$ and $\Hom$.
}.

For completeness, we state without proof that $\cO$ is Artinian. We will not
rely on this result here. For a proof, see~\cite[Section 1.11]{humphreys}.

\section{The Central Character Decomposition}
\label{sec:decomposition}

\subsection{Central Characters}

In the case of finite-dimensional modules, the representation theory is
controlled by the action of the Casimir invariant $\Omega\in Z\fg$. For
modules in $\cO$ the situation is more complicated: we need to consider the
action of $Z\fg$ as a whole.

\begin{dfn}[central character]\label{def:central character}
	A \emph{central character} is an algebra homomorphism \[
		\chi: Z\fg\rightarrow\CC.
	\]
	A module $V$ has central character $\chi$ if $Z\fg$ acts on $V$ by $\chi$.
\end{dfn}

Our aim in this section is to classify the central characters.
We first observe that, while an arbitrary module does not necessarily have a
central character, simple modules do. In particular, for a simple weight module
$L(\lambda)$, we write $\chi_\lambda$ for its central character.

We will work towards the following theorem:

\begin{thm}[Harish-Chandra~\cite{harish-chandra}]\label{thm:harish-chandra}
	Every central character is of the form $\chi_\lambda$ for a weight
	$\lambda\in\fh^*$.
\end{thm}

\begin{proof}
	Our strategy will be to construct a homomorphism $\psi: Z\fg\rightarrow U\fh$,
	called the \emph{twisted Harish-Chandra homomorphism}, which restricts to an
	isomorphism $Z\fg\simeq U\fh^W$, where the latter is the set of $W$-invariants
	in $U\fh$. We can then transport a given central character $\chi$ along this
	isomorphism to obtain an algebra homomorphism $U\fh^W\rightarrow\CC$, which
	extends to a homomorphism $\varphi_\chi: U\fh\rightarrow\CC$.

	Now the algebra $U\fh$ is naturally isomorphic (by taking the double-dual) to
	the algebra of polynomial functions on $\fh^*$. Naturality associates to any
	algebra homomorphism $U\fh\rightarrow\CC$ an element $\lambda\in\fh^*$ which
	represnts the map $\text{ev}_\lambda: f\mapsto f(\lambda)$. However, each such
	$\lambda$ already represents the map $\chi_\lambda$ defined above, yielding the
	desired result. Less abstractly, letting $\lambda$
	represent $\varphi_\chi$ and picking $z\in Z\fg$, we have that \[
		\chi(z) = \varphi_\chi(\psi(z)) = \lambda(\psi(z)) = \chi_\lambda(z).
	\]

	As such, it remains to construct the promised homomorphism $\psi$. [TODO: this]
\end{proof}

% We first observe that the
% kernel of each central character is a maximal ideal, since the image is a field,
% and indeed that this association is bijective.

\subsection{The Decomposition}

Fix a $U\fg$-module $V\in\cO$. Because $Z\fg$ acts locally finitely, we expect
to be able to decompose $V$ into finitely many blocks on which $Z\fg$ acts by a
central character. We explore this decomposition here.

For a central character $\chi$, define \[
	V^\chi = \{v:  \text{for each $z\in Z\fg$}, (z - \chi(z))^l\cdot v =
	0\text{ for some $l$}\}.
\]
We have that $V^\chi$ is a $U\fg$-submodule, since the $z$s commute through the
action of $U\fg$. Furthermore, if $\chi\neq\zeta$, then $V^\chi\cap V^\zeta =
	0$. Now since $Z\fg$ acts locally finitely on the module $V_\mu$, we conclude
that \begin{equation}\label{eqn:partial decomposition}
	V_\mu = \bigoplus_\chi (V^\chi\cap V_\mu),
\end{equation}and so \begin{equation}\label{eqn:module decomposition}
	V = \bigoplus_\chi V^\chi.
\end{equation}Furthermore, this sum is finitely supported, since $V$ is
finitely generated.

We remark that each of the axioms of \Cref{def:category O} are needed here.
In particular, we needed \ref{ax:n-finite} to get that $Z\fg$ acts locally
finitely, which in turn implied \eqref{eqn:partial decomposition}. We needed
\ref{ax:h-semisimple} to deconstruct $V$ as a weight module and conclude
\eqref{eqn:module decomposition}. Finally, we needed \ref{ax:finitely generated}
to conclude that the sum is finitely supported. As such, this decomposition
relies fundamentally on the structure of $\cO$.

Let $\cO_\chi$ be the full\footnote{
	As before, the adjective ``full'' indicates that we keep some objects, and all of
	the morphisms between those objects.
} subcategory of $\cO$ consisting of modules $V$ such
that $V = V^\chi$. Since each module decomposes $V$ into the direct sum of
$V^\chi$s, we have the block decomposition\footnote{The direct sum of
	abelian categories was defined in \Cref{sec:abelian categories}. To apply this
	definition, we do need to ensure that each of the $\cO_\chi$s are
	additive, but additivity is preserved under taking full subcategories, since
	we keep the entire hom-set.} \[
	\cO \simeq \bigoplus_\chi \cO_\chi.
\]Indeed, objects in this category are tuples of $V^\chi$s, and the equivalence
simply interprets these as their direct sum.

	[I would like here to mention something about block decompositions in abelian
		categories more generally. Because we work with higher-dimensional structures,
		the block decomposition is really more like a product decomposition of objects,
		and a direct sum decomposition of morphisms. Any abelian category has such a
		decomposition, but they are usually very artificial---the magic here is that the
		decomposition is entirely controlled by the central characters of simple weight
		modules.]

\section{Translation Functors}
\label{sec:translation functors}

[The plan is to define the translation functor $T^\lambda_\mu$, state without
	proof its relation to the Weyl group (section 7.3 of Humphreys), and give some
	intuition for the theorem in section 7.8, at least in the case of integral
	weights. That theorem states that compatible integral
	weights---those whose difference lies in the root lattice---generate
	categorically equivalent blocks.

	Honestly, depending how many pages it takes to complete the proof of the
	Harish-Chandra theorem, I might not have room/time to include this section at
	all.]

\section{Conclusion}
\label{sec:conclusion}



\nocite{*}
\printbibliography

\end{document}
